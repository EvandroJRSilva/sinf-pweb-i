\documentclass[a4paper, 12pt]{article}
%\documentclass[a4paper,12pt,openany]{memoir}

\input{preambulo.tex}

\title{\vspace{-2cm}\large\textbf{PLANO DE ENSINO}}
\author{}
\date{}

\begin{document}

\maketitle
\thispagestyle{fancy}
%\vspace{-6em}
\vspace{-2.5cm}
    
%-----------------------------------------------------------    
\section{Identificação}
    
\begin{tblr}{vlines, hlines}
    \SetCell[c=2]{wd=0.7\linewidth}Disciplina: Programação para a Web I & & \SetCell{wd=0.25\linewidth}Créditos: 2.2.0 \\
    \SetCell[c=2]{wd=0.7\linewidth}Carga horária: 60 horas & &  \SetCell{wd=0.25\linewidth}Período: 4\textordmasculine
\end{tblr}

%-----------------------------------------------------------
\section{Ementa}

\begin{itemize}
    \item Introdução a WEB.
    \item HTML.
    \item DHTML.
    \item JavaScript.
    \item XML.
    \item CSS.
    \item Páginas Dinâmicas.
    \item Arquiteturas de Desenvolvimento para Web.
    \item Padrões de Projeto para Web.
    \item Cookies.
\end{itemize}  

%-----------------------------------------------------------
\section{Objetivos}

\begin{itemize}
    \item Compreender os princípios básicos da Web, HTML, CSS e estruturação de páginas..
    \item Explorar JavaScript, DHTML, manipulação do DOM e CSS dinâmico.
    \item Trabalhar com XML, cookies, padrões e arquiteturas de desenvolvimento para Web.
\end{itemize}

\newpage
%-----------------------------------------------------------
\section{Conteúdo Programático}

\DeclareTblrTemplate{conthead-text}{default}{ (Continuação)}
\DeclareTblrTemplate{contfoot-text}{default}{Continua na próxima página}
\DeclareTblrTemplate{caption-text}{default}{Conteúdo Programático}
\begin{longtblr}{colspec = {Q[l, 0.8\textwidth] X[c]},
        row{1} = {font=\bfseries, m},
        cells = {m},
        hlines, vlines
        }
    Conteúdo & Carga Horária\\
    \begin{itemize}
        \item Fundamentos da Web
            \begin{itemize}
                \item \textbf{Introdução à Web}: História, funcionamento da Web, cliente-servidor, navegadores, protocolos (HTTP/HTTPS).
                \item \textbf{Fundamentos de HTML}: Estrutura básica de uma página, tags principais (headings, parágrafos, links, imagens).
                \item \textbf{HTML Avançado}: Listas, tabelas, formulários e elementos semânticos.
                \item \textbf{Introdução ao CSS}: Conceitos básicos de estilo, seletores, cores, fontes e unidades de medida.
                \item \textbf{CSS Avançado}: Box model, posicionamento, layouts com flexbox e grid.
                \item Prática.
            \end{itemize}
        \item Avaliação 1
    \end{itemize} & 20\\
    \begin{itemize}
        \item Interatividade e Conteúdo Dinâmico
            \begin{itemize}
                \item \textbf{Introdução ao JavaScript}: Variáveis, operadores, funções e estruturas de decisão.
                \item \textbf{Manipulação do DOM}: Seleção de elementos, eventos, inserção e modificação de conteúdo.
                \item \textbf{DHTML}: Conceitos, integração entre HTML, CSS e JavaScript para páginas dinâmicas.
                \item \textbf{Eventos em JavaScript}: onClick, onChange, onMouseOver, key events, timers.
                \item \textbf{Validação de Formulários com JavaScript}: Boas práticas, mensagens de erro, regex.
                \item Prática.
            \end{itemize}
        \item Avaliação 2
    \end{itemize} & 20\\
    \begin{itemize}
        \item Estruturas Avançadas e Boas Práticas
            \begin{itemize}
                \item \textbf{Introdução ao XML}: Estrutura, tags personalizadas, atributos, bem-formação.
                \item \textbf{Integração de XML}: Utilização em páginas Web, XML vs JSON.
                \item \textbf{Cookies e Armazenamento Local}: Criação, leitura e exclusão de cookies, sessionStorage e localStorage.
                \item \textbf{Páginas Dinâmicas}: Geração de conteúdo dinâmico com scripts e integração de dados.
                \item \textbf{Arquiteturas de Desenvolvimento para Web}: Cliente-servidor, MVC, camadas de aplicação.
                \item \textbf{Padrões de Projeto para Web}: Boas práticas de design, responsividade, acessibilidade.
                \item Prática.
            \end{itemize}
        \item Avaliação 3
    \end{itemize} & 20\\
\end{longtblr}

%-----------------------------------------------------------
\section{Procedimento de Ensino}

O ensino desta disciplina se dará a partir de variados métodos, os quais incluem: 
\begin{itemize}
    \item Aula expositiva, com uso de \textit{slides} e códigos de exemplo;
    \item Atividades práticas
        \begin{itemize}
            \item Trabalhos individuais ou em grupo;
            \item Resolução de exercícios.
        \end{itemize}
\end{itemize}

%-----------------------------------------------------------
\section{Competências e Habilidades}

\begin{itemize}
    \item Competências
        \begin{itemize}
            \item \textbf{Compreensão dos fundamentos da Web}: Entender como funciona a comunicação cliente-servidor, protocolos e a estrutura da Internet.
            \item \textbf{Domínio de tecnologias de marcação e estilo}: Saber estruturar e estilizar documentos HTML e CSS de acordo com padrões modernos.
            \item \textbf{Capacidade de desenvolver interatividade}: Utilizar JavaScript e DHTML para manipular elementos da página e tornar aplicações mais dinâmicas.
            \item \textbf{Uso de boas práticas de desenvolvimento}: Aplicar conceitos de responsividade, acessibilidade e padrões de projeto em aplicações Web.
            \item \textbf{Integração de diferentes tecnologias Web}: Compreender como HTML, CSS, JS, XML, cookies e armazenamento local interagem para formar aplicações funcionais.
            \item \textbf{Pensamento lógico e analítico}: Resolver problemas práticos de programação e estruturar soluções de forma organizada.
            \item \textbf{Trabalho orientado a projetos}: Projetar, implementar e testar aplicações Web simples, aplicando os conceitos estudados.
        \end{itemize}
    \item Habilidades
        \begin{itemize}
            \item Estruturação de páginas Web com HTML, incluindo uso de elementos semânticos, listas, tabelas e formulários.
            \item Estilização de interfaces utilizando CSS, incluindo layout com flexbox, grid e design responsivo.
            \item Manipulação do DOM com JavaScript, alterando conteúdo, estilos e atributos de forma dinâmica.
            \item Uso de eventos em JavaScript para criar interatividade (cliques, teclado, mouse, etc.).
            \item Validação de formulários utilizando expressões regulares e boas práticas de usabilidade.
            \item Criação de páginas dinâmicas integrando HTML, CSS e JavaScript.
            \item Uso de XML e JSON como formatos de armazenamento e troca de dados em aplicações Web.
            \item Manipulação de cookies e armazenamento local para personalização e persistência de dados no navegador.
            \item Aplicação de padrões e boas práticas no desenvolvimento Web, como separação de camadas, design responsivo e acessibilidade.
            \item Desenvolvimento de pequenos projetos Web de forma incremental, consolidando teoria e prática.
        \end{itemize}
\end{itemize}

%-----------------------------------------------------------
\section{Sistemática de Avaliação}

Ao fim de cada unidade, será realizada uma avaliação parcial dos conteúdos ministrados durante o curso da unidade, totalizando em 03 (três) avaliações. A nota de cada avaliação poderá ser composta por um ou mais instrumentos de avaliação, de acordo com um dos seguintes casos: (1) Uma prova escrita; (2) um ou mais trabalhos (individuais ou em grupo); (3) Um ou mais trabalhos, mais uma prova escrita.

Nos casos em que a avaliação for composta por mais de um instrumento, será realizado o somatório ou a média ponderada das notas obtidas em cada instrumento para compor a nota final de uma avaliação parcial. Os instrumentos a serem utilizados em cada avaliação serão definidos e informados no decorrer do curso.

As notas obedecem a uma escala de 0,0 (zero) a 10,0 (dez), contando até a primeira ordem decimal com possíveis arredondamentos. Considerar-se-á aprovado na disciplina o aluno que obtiver assiduidade igual ou superior a 75\% e a média aritmética nas avaliações parciais (média parcial) igual ou superior a 7,0 (sete), ou que se submeta a exame final e obtenha média aritmética entre a média parcial e exame final (média final) igual ou superior a 6,0 (seis). Terá direito de realizar exame final o aluno que satisfaça os requisitos de assiduidade e que obtenha média parcial maior ou igual a 4,0 (quatro) e menor que 7,0 (sete).

A seguir são apresentadas algumas normas, que regulamentam o rendimento escolar
nos Cursos de Graduação da UFPI, aprovados pela resolução no 177/12 de 05/11/2012 do CEPEX/UFPI, atualizada em 03 de maio de 2023:

\vspace{10pt}

\noindent\textbf{Art. 100.} Entende-se por assiduidade do aluno a frequência às atividades didáticas (aulas teóricas e práticas e demais atividades exigidas em cada disciplina) programadas para o período letivo.

\textbf{Parágrafo Único.} Não haverá abono de faltas, ressalvado os casos previstos em legislação específica.

\noindent\textbf{Art. 105.} O professor deve discutir os resultados obtidos em cada instrumento de avaliação junto aos alunos.

\textbf{Parágrafo único.} A discussão referida no caput deste artigo será realizada por ocasião da publicação dos resultados e o aluno terá vista dos instrumentos de avaliação, devendo devolvê-los após o fim da discussão.

\noindent\textbf{Art. 108.} Impedido de participar de qualquer avaliação, o aluno tem direito de requerer a oportunidade de realizá-la em segunda chamada.

\textbf{§ 1º} O aluno poderá requerer exame de segunda chamada por si ou por procurador legalmente constituído. O requerimento dirigido ao professor responsável pela disciplina, devidamente justificado e comprovado, deve ser protocolado à chefia do departamento ou curso a qual o componente curricular esteja vinculada, no prazo de 3 (três) dias úteis, contado este prazo a partir da data da avaliação não realizada.

\textbf{§ 2º} Consideram-se motivos que justificam a ausência do aluno às verificações parciais ou ao exame final:
\begin{enumerate}[label= \alph*)]
    \item doença;
    \item doença ou óbito de familiares diretos;
    \item audiência judicial;
    \item militares, policiais e outros profissionais em missão oficial;
    \item participação em congressos, reuniões oficiais ou eventos culturais
    representando a UFPI, o Município ou o Estado;
    \item outros motivos que, apresentados, possam ser julgados procedentes.
\end{enumerate}

%-----------------------------------------------------------
\section{Bibliografia}

\subsection{Básica}

\begin{itemize}
    \item PILGRIM, Mark. \textbf{Dive into HTML5 with illustrations from the public domain}. Disponível em <\url{https://mislav.github.io/diveintohtml5/}> (HTML, contéudo em Inglês), ou <\url{https://www.jesusda.com/docs/ebooks/ebook_manual_en_dive-into-html5.pdf}> (PDF), ou <\url{https://github.com/zenorocha/diveintohtml5}> (conteúdo traduzido). Acesso a todas as páginas citadas em 20 ago. 2025.
    \item \textbf{Aprenda Layout com CSS}. Disponível em <\url{https://pt-br.learnlayout.com/}>. Acesso em 20 ago. 2025.
    \item HAVERBEKE, Marijn. \textbf{Eloquent JavaScript}. 4. ed. No Starch Press, 2024. Disponível em <\url{https://eloquentjavascript.net/}>. Acesso em 20 ago. 2025.
    \item DEITEL, H. M; DEITEL, P. J. et. al. **Java TM: como programar**. 8 ed. São Paulo: Pearson, 2010.
    \item CRANE, D.; PASCARELLO, E.; DARREN, J. **Ajax em Ação**. 1 ed. São Paulo: Pearson, 2007.
    \item HORTMAN, C. S.; CORNELL, G. **Core Java: Volume 1**. 8 ed. São Paulo: Pearson, 2010.
\end{itemize}

\subsection{Complementar}

\begin{itemize}
    \item DUCKETT, Jon. \textbf{HTML and CSS: Design and Build Websites}. 1 ed. Wiley, 2014.
    \item OGDEN, Max. \textbf{JavaScript For Cats}. Disponível em <\url{https://jsforcats.com/}>. Acesso em 20 ago. 2025.
    \item \textbf{W3Schools Tutorials} (HTML, CSS, JavaScript, XML, Cookies). Disponível em <\url{https://www.w3schools.com/}>. Acesso em 20 ago. 2025.
    \item HORTMAN, C. S.; CORNELL, G. \textbf{Core Java: Volume 2}. 8 ed. São Paulo: Pearson, 2010.
    \item BURKE, B; MONSON, R. \textbf{Enterprise JavaBeans 3.0}. 5 ed. São Paulo: Peason, 2007.
    \item JENDROCK, E; GOLLAPUDI, H; SRIVATHSA. \textbf{JAVA EE 6 Tutorial: The Basic Concepts}. 4 ed. São Paulo: Pearson, 2011.
    \item FELKE-MORRIS, T. \textbf{Basic of Web HTML5 Design \& CSS3}. 1 ed. São Paulo: Prentice-hall, 2012.
    \item GRAHAM, S. \textbf{Building WebService with Java}. 2 ed. São Paulo: Peason, 2005.
\end{itemize}


\vfill
%-----------------------------------------------------------
% ASSINATURAS
\begin{center}
    \rule{6cm}{0.4pt} \\ 
    \textbf{Evandro José da Rocha e Silva} \\
    Professor(a) do Curso de Sistemas de Informação \\[1.5cm]
    
    \rule{6cm}{0.4pt} \\ 
    \textbf{Frank César Lopes Véras} \\
    Professor e Coordenador do Curso de Sistemas de Informação
\end{center}
    
\end{document}